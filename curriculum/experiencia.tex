\cvsection{Experiência}
\begin{cventries}
  \cventry
    {SRE - Sênior}
    {iFood}
    {Remoto}
    {Dez. 2022 - Agora}
    {
      \begin{cvitems}
        \item \textit{\textbf{No time de traffic/network-l7}}
        \begin{itemize}
          \item Manutenção e evolução da camada de borda e API Gateways dos serviços
          \item Desenvolvimento de aplicações para melhorar a experiencia do desenvolvedor
          \item Evolução da arquitetura das ferramentas internas para garantir resiliencia
          \item Oncall
        \end{itemize}
      \end{cvitems}
    }
  
  \cventry
    {Consultor de Infraestrutura - Sênior}
    {Thoughtworks}
    {Porto Alegre, RS}
    {Abr. 2020 - Dez. 2022}
    {
      \begin{cvitems}
        \item \textit{\textbf{Como consultor em uma multinacional do ramo de alimentos}}
        \begin{itemize}
            \item Contrução de pipelines para os projetos no Gitlab CI, com deploy em Cloud Functions e Kubernetes Engine, e exposição no Apigee.
            \item Desenho de arquitetura e provisionamento de infraestrutura base para um app de food delivery, no Google Cloud Platform
        \end{itemize}
        \item \textit{\textbf{Como consultor em uma multinacional do ramo de cosméticos}}
        \begin{itemize}
            \item Atuação como Devops Lead, apoiando todos os devops de uma tribo por \char`\~ 4 meses
            \item Auxilio na implantação do AWS Cognito como sistema de autenticação padrão das marcas da compania
            \item Auxilio na migração entre API Gateways, do Kong para o API Gateway da AWS
            \item Criação de módulos para plataforma interna de developer experience, possibilitando "self-service" de serviços na cloud
            \item Melhorias no portal oferecido para consultoras filiadas, migrando partes monolíticas para microfrontends e BFFs específicos
            \item Disponiblização de landing pages para atração de novas consultoras, em arquiteturas com CDN
            \item Evolução do sistema de auto-cadastro de novas consultoras, e adaptação do mesmo para utilização em outros países
        \end{itemize}
      \end{cvitems}
    }
    
  \cventry
    {Analista de Infraestrutura}
    {Unicred do Brasil}
    {Porto Alegre, RS}
    {Ago. 2019 - Abr. 2020}
    {
      \begin{cvitems}
        \item \textit{\textbf{Manutenção de um ambiente de microserviços em Java e Kubernetes}}
        \item \textit{\textbf{Implantação do AppDynamics APM}}
        \begin{itemize}
            \item Instalação de agentes e configuração da plataforma para envio de métricas
            \item Configuração da controladora para tracking das business transactions mais importantes
        \end{itemize}
        \item \textit{\textbf{Implantação de produtos da RedHat}}
        \begin{itemize}
            \item Auxílio na implantação do RHEV - RedHat Enterprise Virtualization, Satellite e Insights
            \item Migração de algumas VMs para o novo ambiente
        \end{itemize}
        \item \textit{\textbf{Iniciar projeto na nuvem}}
        \begin{itemize}
            \item Auxiliar desenvolvedores trazendo uma mentalidade de cloud-native para seus aplicativos.
            \item Provisionamento de ambientes novos necessários na AWS.
            \item Adaptar comunicação com serviços internos, que não estão na núvem via VPN
        \end{itemize}
      \end{cvitems}
    }
    
  \cventry
    {DevOps Engineer}
    {Umbler Inc.}
    {Gravataí, RS}
    {Fev. 2018 - Ago. 2019}
    {
      \begin{cvitems}
        \item \textit{\textbf{Arquitetura de melhorias para a plataforma de hospedagem em container}}
        \item \textit{\textbf{Migração do produto para a AWS}}
        \begin{itemize}
            \item Utilização  de componentes básicos da AWS - com intuito de não ficar "preso" a mesma cloud provider
            \item Automatização de criação de VMs com Terraform e instalação/configuração com Ansible
            \item Auxilio na nudança de mindset: Infraestrutura própria "abundante" vs AWS "enxuto"
            \item Mudança nos componentes do produto para que se adaptem, como o monitoramento, por exemplo.
        \end{itemize}
        \item \textit{\textbf{Migração do produto para o GCP}}
        \begin{itemize}
            \item Assim como foi feito na AWS, o mesmo foi feito para o GCP - por trazer um custo menor nos componentes utilizados e ter a validação de não estar preso a um cloud provider
            \item Ferramentas utilizadas foram as mesmas. Scripts adaptados para provisionamento no GCP
        \end{itemize}
        \item \textit{\textbf{Automatização e migração para o Openstack}}
        \begin{itemize}
            \item Depois de termos testado o produto oferecendo na AWS e no GCP, a decidimos utilizar servidores próprios 
            \item A implantação do OpenStack foi feita sob o virtualizador já existente - Hyper-V
            \item Utilizamos ferramentas adicionais para controle de jobs de automação - como Rundeck
            \item Dessa vez, a automatização foi para níveis maiores: até com a mudança de mindset da empresa para poder crescer exponencialmente
        \end{itemize}
      \end{cvitems}
    }
    
  \cventry
    {Senior PHP Developer}
    {Moovin - Plataforma E-commerce}
    {Porto Alegre, RS}
    {Mar. 2013 - Fev. 2018}
    {
      \begin{cvitems}
        \item \textit{\textbf{Migração da infraestrutura para a AWS}}
        \begin{itemize}
            \item Refatoração do código para torná-lo mais "Cloud ready"
            \item Utilização de cache e sessão: ElastiCache (redis e memcached)
            \item Catálogo de produtos e logs: AWS Elasticsearch
            \item Tarefas pontuais e redimensionamento de imagem: S3/Lambda
            \item Tratamento de emails: SES, Kinesis, AWS Elasticsearch
            \item Utilização geral: EC2, RDS, ELB, CloudFront
        \end{itemize}
        \item \textit{\textbf{Desenvolvimento do CI/CD}}
        \begin{itemize}
            \item Utilização do Jenkins e suas pipelines com script e declarativas
            \item Desenvolvimento da solução de teste
            \item Deploy para produção cerca 8 vezes ao dia
        \end{itemize}
        \item \textit{\textbf{Refatoração de código monolítico e containerização de partes da plataforma}}
        \item \textit{\textbf{Implementação da "nova plataforma"}}
        \begin{itemize}
            \item Desenvolvimento de novos microsserviços
            \item Disponibilização e manutenção de um cluster Kubernetes (na AWS)
            \item Configuração de pacotes Helm para cada projeto
        \end{itemize}
        \item \textit{\textbf{Migração do cluster de Kubernetes da AWS para o GCP}}
        \begin{itemize}
            \item Devido ao custo elevado com o HA de Managers no Kubernetes optamos por migrar para o GCP - que dava isso de graça
            \item Port dos serviços de LoadBalancer e storage stático (S3)
        \end{itemize}
      \end{cvitems}
    }

\end{cventries}
